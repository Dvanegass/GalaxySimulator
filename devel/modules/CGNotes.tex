% Copyright (c) 2012-2022 by the GalSim developers team on GitHub
% https://github.com/GalSim-developers
%
% This file is part of GalSim: The modular galaxy image simulation toolkit.
% https://github.com/GalSim-developers/GalSim
%
% GalSim is free software: redistribution and use in source and binary forms,
% with or without modification, are permitted provided that the following
% conditions are met:
%
% 1. Redistributions of source code must retain the above copyright notice, this
%    list of conditions, and the disclaimer given in the accompanying LICENSE
%    file.
% 2. Redistributions in binary form must reproduce the above copyright notice,
%    this list of conditions, and the disclaimer given in the documentation
%    and/or other materials provided with the distribution.
%

\documentclass{article}
\usepackage[utf8]{inputenc}
\usepackage{geometry}
\usepackage[usenames,dvipsnames,svgnames,table]{xcolor}
%\usepackage{color}
\usepackage{graphicx}
\usepackage{amsmath}
\usepackage{amsfonts}
\usepackage{mathabx}
\usepackage{commath}

\geometry{textwidth=6.5in, textheight=9.0in,
    marginparsep=7pt, marginparwidth=.6in}
\setlength{\parindent}{0in}
\setlength{\parskip}{0.08in}

\newcommand{\red}[1]{\textcolor{red}{#1}}
\newcommand{\blue}[1]{\textcolor{blue}{#1}}
\newcommand*{\annot}[1]{\tag*{\footnotesize{\textcolor{gray}{#1}}}}

\let\Re\undefined
\DeclareMathOperator{\Re}{Re}
\let\Im\undefined
\DeclareMathOperator{\Im}{Im}

\title{Color Gradients Notes}
\author{Josh Meyers}
\date{October 2015}

\begin{document}

\section{Introduction}

The goal is to create a \textsc{GalSim} ChromaticObject with realistic color gradients using high
resolution multiband images (most likely from HST) as input.  This is done by modeling the
preconvolution chromatic surface brightness profile $f(\vec{x}, \lambda)$ as a sum of two or more
separable chromatic surface brightness profiles, each with a particular asserted SED.  I.e.,

\begin{equation}
    \label{eqn:sbprof}
    f(\vec{x}, \lambda) = \sum_j S_j(\lambda) a_j(\vec{x}),
\end{equation}

where $S_j(\lambda)$ is the $j$th SED asserted as part of the decomposition, and $a_j(\vec{x})$ is
the spatial component of the $j$th separable chromatic profile.

The required set of inputs is:

\begin{itemize}

\item Two or more HST images of the same galaxy in different filters, $I_i(\vec{x})$, where $i$
labels the different filters.

\item The 2D noise covariance function for each image, $\xi_i(\Delta\vec{x})$.  If the noise is
uncorrelated, then this could be simply the noise variance ($\xi_i(\Delta\vec{x}=0) = \sigma^2_i$, $\xi_i(\Delta\vec{x}\ne0)=0$).

\item The chromatic HST PSF, $\Pi(\vec{x}, \lambda)$.

\item The HST throughput for each image filter, $T_i(\lambda)$.

\item Two or more SEDs, $S_j(\lambda)$, to use in the decomposition in Equation \ref{eqn:sbprof}.

\end{itemize}

The model for the observed images is:

\begin{align}
    I_i(\vec{x})
    &= \int T_i(\lambda) \left[\Pi(\vec{x}, \lambda) \Asterisk f(\vec{x}, \lambda)\right] \dif{\lambda} + \eta_i(\vec{x}) \\
    &= \int T_i(\lambda) \sum_j S_j(\lambda) \left[\Pi(\vec{x}, \lambda) \Asterisk a_j(\vec{x})\right] \dif{\lambda} + \eta_i(\vec{x})
\end{align}

where $\eta_i(\vec{x})$ indicates (potentially spatially correlated) Gaussian noise in image $i$ and
the $\Asterisk$ symbol indicates (spatial) convolution.  The noise $\eta_i(\vec{x})$ is related to
the noise covariance function via $\langle\eta_i(\vec{x}_l) \eta_i(\vec{x}_m)\rangle =
\xi_i(\vec{x}_l - \vec{x}_m)$, where angle brackets indicate averaging over realizations of the
noise.  Note that we also assume that $\xi_i$ is even: $\xi_i(\Delta\vec{x}) =
\xi_i(-\Delta\vec{x})$.

The convolution is easier to work with in Fourier space where it becomes a mode-by-mode product.
Indicating the Fourier transform of the real-space quantity $g(\vec{x})$ with $\tilde{g}(\vec{k})$,
the model in Fourier space is

\begin{align}
    \tilde{I}_i(\vec{k})
    &= \int T_i(\lambda) \sum_j S_j(\lambda) \tilde{\Pi}(\vec{k}, \lambda) \tilde{a}_j(\vec{k}) \dif{\lambda} + \tilde{\eta}(\vec{k}) \\
    &= \sum_j \left[\int T_i(\lambda) S_j(\lambda) \tilde{\Pi}(\vec{k}, \lambda) \dif{\lambda}\right] \tilde{a}_j(\vec{k}) + \tilde{\eta}(\vec{k}) \\
    \label{eqn:solveme}
    &=  \sum_j \tilde{\Pi}^\mathrm{eff}_{ij}(\vec{k}) \tilde{a}_j(\vec{k}) + \tilde{\eta}(\vec{k}),
\end{align}

where the effective PSF for the $i$th filter and $j$th SED is

\begin{equation}
  \tilde{\Pi}^\mathrm{eff}_{ij}(\vec{k}) = \int T_i(\lambda) S_j(\lambda) \tilde{\Pi}(\vec{k}, \lambda) \dif{\lambda}.
\end{equation}

The crux of the problem is to solve for the (complex-valued) $\tilde{a}_j(\vec{k})$ and to propagate
the statistics of the noise.

\section{Solving for $\tilde{a}_j(\vec{k})$}

There are several possible ways to estimate the required $\tilde{a}_j(\vec{k})$.

\subsection{No noise}

We'll look first at the case where one ignores noise entirely ($\eta(\vec{x}) =
\tilde{\eta}(\vec{k})= 0$).

If the number of asserted SEDs $N_j$ is larger than the number of input images $N_i$, then the
system of equations represented by Equation \ref{eqn:solveme} is underdetermined.  While in
principle this could still be solvable using some prior constraints, for now, we'll just ignore this
possibility.

If $N_j = N_i$, then we can exactly solve Equation \ref{eqn:solveme} using matrix inversion for each
Fourier mode $\vec{k}$:

\begin{equation}
  \tilde{a}_j(\vec{k}) = \sum_i [(\tilde{\Pi}^\mathrm{eff}(\vec{k}))^{-1}]_{ij} \tilde{I}_i(\vec{k}).
\end{equation}

If there are more images than spectra ($N_i > N_j$), then the system of equations is overdetermined
and no longer has an exact solution.  Instead, there is a (usually) unique ``least-squares''
solution for each $\vec{k}$ obtained by minimizing

\begin{equation}
  \label{eqn:lstsq}
  \sum_i\left|\tilde{I}_i(\vec{k}) - \sum_j \tilde{\Pi}^\mathrm{eff}_{ij}(\vec{k}) \tilde{a}_j(\vec{k})\right|^2.
\end{equation}

\subsection{Uncorrelated noise}

Stationary, uncorrelated Gaussian pixel noise in the $i$th image is completely described by its
variance $\sigma^2_i$.  In Fourier space, the variance of each mode is independent of $\vec{k}$ and
proportional to $\sigma^2_i$, where the constant of proportionality depends on the particular
Fourier conventions employed.  We can therefore write down a likelihood for $\tilde{a}_j(\vec{k})$
as

\begin{equation}
    \label{eqn:like}
    \chi^2(\vec{k}) = -2 \log \mathcal{L}(\vec{k}) = \sum_i\frac{1}{\sigma_i^2}\left|\tilde{I}_i(\vec{k}) - \sum_j \tilde{\Pi}^\mathrm{eff}_{ij}(\vec{k}) \tilde{a}_j(\vec{k})\right|^2.
\end{equation}

This is a \textit{weighted} least-squares problem.  Note that the weights only matter for
determining the Fourier coefficients $\tilde{a}_j(\vec{k})$ if the number of input images is greater
than the number of asserted spectra.  When these quantities are equal, it is always possible to find
$\tilde{a}_j(\vec{k})$ such that $\tilde{I}_i(\vec{k}) = \sum_j
\tilde{\Pi}^\mathrm{eff}_{ij}(\vec{k}) \tilde{a}_j(\vec{k})$.  The weights do still matter in this
case if we're interested in propagating the noise, however.


\subsection{Correlated noise}

If the noise covariance function for a particular image $\xi_i(\vec{\Delta x})$ is non-zero away
from the origin, then the variance of different Fourier modes is not constant, but proportional to
the noise power spectrum, which is the Fourier transform of the noise covariance function:

\begin{equation}
  P_i(\vec{k}) = \int \xi_i(\Delta \vec{x}) e^{-2 \pi i \vec{k}\cdot\Delta\vec{x}}.
\end{equation}

Due to assumed translation invariance of the noise (though note that we do not assume isotropy), the
covariance of different Fourier modes vanishes.  More explicitly, if the noise in image $i$ and mode
$\vec{k_l}$ is $\tilde\eta_i(\vec{k_l})$, then

\begin{equation}
  \langle\tilde\eta_i^*(\vec{k_l})\tilde\eta_i(\vec{k_m})\rangle = \delta(\vec{k_l} - \vec{k_m})P_i(\vec{k_l}).
\end{equation}

Once the noise power spectrum has been computed, therefore, the only change to the likelihood in
Equation \ref{eqn:like} is that the mode variance now depends on the particular Fourier mode in
question:

\begin{equation}
    \label{eqn:likecorr}
    -2 \log \mathcal{L}(\vec{k}) = \sum_i\frac{1}{P_i(\vec{k})}\left|\tilde{I}_i(\vec{k}) - \sum_j \tilde{\Pi}^\mathrm{eff}_{ij}(\vec{k}) \tilde{a}_j(\vec{k})\right|^2.
\end{equation}

This is essentially the same weighted least squares problem as in the uncorrelated noise case -- the
only difference being that the weights now depend on $\vec{k}$.  Written in matrix notation, the
solution for a particular $\vec{k}$ mode is:

\begin{equation}
    \tilde{a} = \left(\tilde{\Pi}^{\mathrm{eff}, \dagger} W \tilde{\Pi}^\mathrm{eff} \right)^{-1} \tilde{\Pi}^{\mathrm{eff}, \dagger} W \tilde{I}
\end{equation}

where
\begin{equation}
    W_{ij} = \delta_{ij}/P_i.
\end{equation}

The covariance matrix for the elements of $\tilde{a}$ is given by
\begin{equation}
    \Sigma = \left(\tilde{\Pi}^{\mathrm{eff}, \dagger} W \tilde{\Pi}^\mathrm{eff} \right)^{-1}.
\end{equation}

\section{Propagating the noise covariance}

At this point, we have the $\tilde{a}_j(\vec{k})$ necessary to represent a chromatic surface
brightness profile in Fourier space as

\begin{equation}
  \tilde{f}(\vec{k}, \lambda) = \sum_j S_j(\lambda) \tilde{a}_j(\vec{k}).
\end{equation}

The uncertainty in each $\tilde{a}_j(\vec{k})$ is uncorrelated from one $\vec{k}$ to the next (since
we defined independent likelihoods for each $\vec{k}$), but does, in general, possess correlations
between the different SED components $j$ and $j^\prime$ for a given $\vec{k}$.  Using
$\Sigma_{jj^\prime}(\vec{k})$ to represent these covariances, we have

\begin{equation}
    \langle\delta\tilde{a}^*_j(\vec{k}_l)\delta\tilde{a}_{j^\prime}(\vec{k}_m)\rangle = \delta(\vec{k}_l - \vec{k}_m)\Sigma_{jj^\prime}(\vec{k_l}).
\end{equation}

Fortunately, $\Sigma_{jj^\prime}(\vec{k})$ is analytically computable for weighted least squares
problems.

This $\Sigma_{jj^\prime}(\vec{k})$ can then be transformed alongside the $\tilde{a}_j(\vec{k})$ to
effect shears, rotations, dilations, etc., the same way that noise is transformed in existing
\textsc{GalSim} routines.

Finally, the model for creating a (Fourier-domain) output image $\tilde{I}^\mathrm{out}(\vec{k})$,
convolving by an output chromatic PSF $\tilde{\Pi}^\mathrm{out}(\vec{k}, \lambda)$ and drawing
through an output filter with transmission $T^\mathrm{out}(\lambda)$, is

\begin{align}
    \tilde{I}^\mathrm{out}(\vec{k})
    &= \int T^\mathrm{out}(\lambda) \tilde{\Pi}^\mathrm{out}(\vec{k}, \lambda) \tilde{f}(\vec{k}, \lambda) \dif{\lambda} \\
    &= \int T^\mathrm{out}(\lambda) \tilde{\Pi}^\mathrm{out}(\vec{k}, \lambda) \sum_j S_j(\lambda) \tilde{a}_j(\vec{k}) \dif{\lambda} \\
    &= \sum_j \tilde{\Pi}_j^\mathrm{out, eff}(\vec{k}) \tilde{a}_j(\vec{k})
\end{align}

where the effective PSF for the $j$th component of the output image is

\begin{equation}
  \tilde{\Pi}_j^\mathrm{out, eff}(\vec{k}) = \int \tilde{\Pi}^\mathrm{out}(\vec{k}, \lambda) T^\mathrm{out}(\lambda) S_j(\lambda)\dif{\lambda}.
\end{equation}

Propagating the (potentially transformed) covariance spectrum $\Sigma_{jj^\prime}(\vec{k})$ into the
final output noise power spectrum $P^\mathrm{out}(\vec{k})$ then follows the normal propagation of
errors formula:

\begin{equation}
  P^\mathrm{out}(\vec{k}) = \sum_{jj^\prime} \tilde{\Pi}_j^\mathrm{out, eff, *}(\vec{k}) \Sigma_{jj^\prime}(\vec{k}) \tilde{\Pi}_{j^\prime}^\mathrm{out, eff}(\vec{k}).
\end{equation}

This power spectrum can then be used in the existing correlated noise whitening and symmetrizing
GalSim routines.

\section{Implementation notes}

\textsc{GalSim} uses the methodology developed by Bernstein \& Gruen (2013) to interpolate
discretely sampled surface brightness profiles in real and Fourier space.  Here we investigate the
impact of this interpolation on the Wiener-Khinchin theorem, which relates the real-space
autocorrelation function to the Fourier-space power spectrum.

The discrete version of the Wiener-Khinchin theorem is derived as follows.  We are interested in
the (co)-variance of the Discrete Fourier Transform amplitudes of some discrete real space samples
with discrete auto-covariance function $\xi\left[r\right]$ (we follow the convention of
Rowe++14 in using square brackets to indicate the arguments of a discretly sampled function, and
reserve parentheses to indicate arguments of continuous objects).  Note that since we're using DFTs,
$\xi$ is implicitly periodic: $\xi\left[r\right] = \xi\left[r+N\right]$.

The 1D DFT of an N-point sampled function is:

\begin{equation}
    \tilde{b}_k = \sum_{j=-N/2}^{N/2-1} b_j e^{-2 \pi i j k / N}.
\end{equation}

% with inverse transform
%
% \begin{equation}
%     b_j = \frac{1}{N}\sum_{k=-N/2}^{N/2-1} \tilde{b}_k e^{2 \pi i j k / N}.
% \end{equation}

We are interested in the quantity $\left\langle\tilde{b}^*_k \tilde{b}_{k^\prime}\right\rangle$
where the angle brackets indicate averaging over noise realizations.

\begin{align}
    \left\langle\tilde{b}^*_k \tilde{b}_{k^\prime}\right\rangle
    \annot{Sub in FT definition}
    &= \left\langle\sum_{j=-N/2}^{N/2-1} b_j e^{+2 \pi i j k / N} \sum_{j^\prime=-N/2}^{N/2-1} b_{j^\prime} e^{-2 \pi i j^\prime k^\prime / N}\right\rangle \\
    \annot{Interchange expectation and summation}
    &= \sum_{j=-N/2}^{N/2-1} \sum_{j^\prime=-N/2}^{N/2-1} \left\langle b_j b_{j^\prime} \right\rangle e^{2 \pi i (jk - j^\prime k^\prime)/N} \\
    \annot{$\xi\left[r\right]$ definition}
    &= \sum_{j=-N/2}^{N/2-1} \sum_{j^\prime=-N/2}^{N/2-1} \xi\left[j-j^\prime\right] e^{2 \pi i (jk - j^\prime k^\prime)/N} \\
    \annot{change variables $r=j^\prime-j$}
    &= \sum_{j=-N/2}^{N/2-1} \sum_{r=-N/2-j}^{N/2-1-j} \xi\left[r\right] e^{2 \pi i (jk - (r+j) k^\prime)/N} \\
    \annot{simplify using periodicty of $\xi$ and $\exp$}
    &= \sum_{j=-N/2}^{N/2-1} \sum_{r=-N/2}^{N/2-1} \xi\left[r\right] e^{-2 \pi i r k^\prime/N} e^{2 \pi i j (k-k^\prime)/N} \\
    \annot{definition of $P\left[k\right]$}
    &= P[k^\prime] \sum_{j=-N/2}^{N/2-1} e^{2 \pi i j (k-k^\prime)/N} \\
    \annot{$e^{2 \pi i j (k-k^\prime)/N}$ evenly samples unit circle unless $k-k^\prime \in N\mathbb{Z}$...}
    &= N P[k^\prime] \sum_{N=0}^\infty\delta^{k+N}_{k^\prime} \\
    \annot{...but we really only care about $k \in \left[-N/2, N/2-1 \right]$ }
    &= N P[k^\prime] \delta^{k}_{k^\prime}.
\end{align}

Now we try the same thing using the Bernstein \& Gruen continuously interpolated Fourier transform.
Recall their result:

\begin{align}
    \tilde{F}(u)
    &= \int{F(x) e^{-2 \pi i u x} \dif{x}} \\
    &\approx \tilde{K}_x(u) \sum_{k=-N/2}^{N/2-1}\tilde{b}_k K_u(u-k/N)
\end{align}

where $K_x$ is an asserted real-space interpolant, $\tilde{K}_x$ is its Fourier transform,
and $K_u$ is an asserted Fourier-space interpolant.  Note that the final equality is exact if and
only if

\begin{equation}
    K_u(v) = e^{i \pi v} \frac{\mathrm{sinc}\, N v}{\mathrm{sinc}\, v}.
\end{equation}

We are interested in the (co)-variance of Fourier amplitudes:

\begin{align}
    \left\langle \tilde{F}^*(u) \tilde{F}(u^\prime)\right\rangle
    \annot{Sub in GalSim approximate FT definition}
    &= \left\langle \tilde{K}_x^*(u) \sum_{k=-N/2}^{N/2-1}\tilde{b}_k^* K_u^*(u-k/N) \tilde{K}_x(u^\prime) \sum_{k^\prime=-N/2}^{N/2-1}\tilde{b}_{k^\prime} K_u(u^\prime-k^\prime/N)  \right\rangle \\
    \annot{Rearrange}
    &= \tilde{K}_x^*(u) \tilde{K}_x(u^\prime) \sum_{k=-N/2}^{N/2-1} \sum_{k^\prime=-N/2}^{N/2-1} K_u^*(u-k/N) K_u(u^\prime-k^\prime/N) \left\langle \tilde{b}_k^* \tilde{b}_{k^\prime} \right\rangle \\
    \annot{definition of $P\left[k\right]$}
    &= \tilde{K}_x^*(u) \tilde{K}_x(u^\prime) \sum_{k=-N/2}^{N/2-1} \sum_{k^\prime=-N/2}^{N/2-1} K_u^*(u-k/N) K_u(u^\prime-k^\prime/N) N P[k^\prime]\delta_{k^\prime}^k \\
    \annot{sum over Kronecker $\delta$}
    &= N \tilde{K}_x^*(u) \tilde{K}_x(u^\prime) \sum_{k=-N/2}^{N/2-1} K_u^*(u-k/N) K_u(u^\prime-k/N) P[k]
\end{align}

This is as far as I've been able to push (ran out of $\delta$ functions!).  Assuming there are no
further simplifications, one implication is that Fourier amplitudes of an interpolated noise image
are not uncorrelated.  Equivalently, the continuous auto-covariance function of interpolated
discretely stationary noise is not itself stationary.  This makes some intuitive sense, I think.
For example, the variance of interpolated points seems like it should always be less than the
variance of the input samples, implying that the continuously regarded noise is not stationary.

\end{document}
